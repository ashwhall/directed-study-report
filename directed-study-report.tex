\documentclass{report}
\usepackage[a4paper]{geometry}
\usepackage{amsmath}
\usepackage{bm}
\usepackage{pgfplots}
\usepackage{ amssymb }
\usepackage{color, soul}
\usepackage[backend=biber]{biblatex}
\addbibresource{references.bib}
\graphicspath{{images/}}


%opening
\title{Directed Study Report\\CSE4DIR}
\author{Ash Hall\\17756156}

\newcommand{\TODO}[1]{\sethlcolor{pink}\hl{\\(#1)\\}}
\newcommand{\FEEDBACK}[1]{\sethlcolor{green}\hl{\\ Feedback: \\#1\\}}
\newcommand{\TOCITE}[2][citation needed]{\textsuperscript{\underline{#1}}}

\begin{document}

	\maketitle
	\thispagestyle{empty}
	\newpage
	\thispagestyle{empty}
	\tableofcontents
	\newpage
	\thispagestyle{empty}
	\newpage
	
	\setcounter{chapter}{1}	
	\chapter*{Directed Study Report}

	\section{Introduction}
	This report is about the effects of catastrophic interference in neural networks. \\
	We'll discuss some techniques for mitigating these effects, and will explore the severity of each.
	The benefit of exploring these things. \\
	
	
	\section{Transfer Learning}
	What transfer learning is \\
	Why we want transfer learning (consider both changing datasets and extending classes) \\
	How transfer learning works (with diagram) \\
	What's wrong with transfer learning (need to keep old examples if extending classes) \\
	
	\section{Continuous Learning}
	What continuous learning is. \\
	Why continuous learning is different to transfer learning (don't look at old examples) \\
	Briefly explain some techniques \\

	\subsection{Related Works}
	Intro to related works \\
	A few related works that address catastrophic interference	with continuous learning, why they're good and bad. \\
	Summary of related works \\

	\subsection{Environment}
	State the environment - (Ubuntu, docker, tensorflow etc.)
	
	\section{Framework}
	A description of the framework heirarchy. I'll give a simple diagram of the file-structure/python modules in the project, then explain roughly what everything is responsible for.
	
	\section{Training Procedure and Metrics}
	Here I'll explain how I trained each of the models (the stuff that's consistent between them), and give an introduction to some terms, and explain the metrics I used.
	
	\section{Benchmark Experiments and Results}
	I'll put a short introduction to the results here,
	
	\subsection{Experiment 1}
	(These will actually have names, not just experiment 1, 2, ...)
	\subsection{Experiment 2}
	\subsection{Experiment 3}
		
	\section{Summary}
	Summarise the experiment results and what impact they have on my project, summarise the work that I've done and where it leaves my with regards to the second semester and the "actual work". 
	
	
\end{document}
